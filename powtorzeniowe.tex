\documentclass[a4paper,fleqn]{article}
\usepackage[utf8]{inputenc}
\usepackage[polish]{babel}
\usepackage{polski}
\usepackage{amsmath}
\usepackage{amssymb}

\usepackage{graphicx}
\usepackage{wrapfig}

\newcommand{\selim}[1][n]{\lim_{#1\to\infty}}
\newcommand{\Rls}{\mathbb{R}}
\newcommand{\Nats}{\mathbb{N}}
\newcommand{\Calk}{\mathbb{Z}}
\newcommand{\Complex}{\mathbb{C}}
\newcommand{\nNat}{n \in \Nats}
\newcommand{\xRl}{x \in \Rls}
\newcommand{\allNats}[1][n]{\forall_{#1 \in \Nats}}
\newcommand{\allRls}[1][x]{\forall_{#1 \in \Rls}}
\newcommand{\dla}{\text{ dla }}
\newcommand{\ddn}[1][n]{\text{ d.d.}#1\text{: }}
\renewcommand{\leq}{\leqslant}
\renewcommand{\geq}{\geqslant}
\newcommand{\sgn}{\operatorname{sgn}}
\newcommand{\im}{\operatorname{Im}}


\usepackage{fullpage}
\title{Zadania z zestawu powtórzeniowego}
\author{}
\date{\today}
\begin{document}
	\section*{Zad. 1}
	Znaleźć i naszkicować
	\paragraph{a}
	\[ \{ z\in\Complex : z^4 = (1+2j)^8 \} \]
	\[ (1+2j)(1+2j) = -3 + 4j \]
	\[ z^4 = (-3+4j)^4 \]
	Pierwiastki na okręgu: $-3+4j$, $4+3j$, $3-4j$, $-4-3j$
	\section*{Zad. 55}
	Dane jest przekształcenie liniowe L, t. że $M_{\mathfrak{A}}^{\mathfrak{A}}(L) = A$.
	Znaleźć bazę $B$, by L miało w tej bazie macierz w postaci kanonicznej Jordana, jeśli:
	\paragraph{a}
	Wielomian charakterystyczny:
	\[ \det(A-\lambda E) = \det \begin{bmatrix}
		2-\lambda & -2 & -1 & 3\\
		0 & 3-\lambda & 1 &2 \\
		0&1 & 5-\lambda & 4 \\
		0 & -1 & -2 &  -1-\lambda
		\end{bmatrix} = \ldots = (\lambda-2)^3 (\lambda-3) \]
	\[ \lambda =2 \lor \lambda = 3 \]

	\[ A-3E = \begin{bmatrix}
		-1 & -2 & -1 & 3 \\
		0&0&1&2\\
		0&1&2&4\\
		0&-1&-2&-4
		\end{bmatrix} \]
	\[ r(A-3E) = 3 \]
	\[ 4-3 = 1 \]
	wektor własny dla $\lambda = 3 $

	\[ \lambda = 2 \]
	\[ A-2E = \begin{bmatrix}
		0 & -2 & -1 & 3 \\
		0& 1 & 1 & 2 \\
		0 & 1 & 3 & 4 \\
		0 & -1 & -2 & -3
		\end{bmatrix}
		\]
	\[ 4 -3 = 1 \]
	\[ r = 3 \]

	wektor dołączony?
	\[ (A-2E)^2 =
		\begin{bmatrix}
			0 & -6 & -11 & -17\\
			0 & 0 & 0 & 0\\
			0 & 0 & 2 & 2\\
			0 & 0 & -1 & -1
		\end{bmatrix} \]
	\[ r((A-2E)^2) = 2 \]
	$ 4 - 2 = 2 \Rightarrow $ 1 wektor dołączony 1szego rzędu.

	\[ (A-2E)^3 =
		\begin{bmatrix}
			0 & 0 & -5 & -5\\
			0 & 0 & 0 & 0\\
			0 & 0 & 2 & 2\\
			0 & 0 & -1 & -1
		\end{bmatrix}
	\]
	\[ r((A-2E)^3) = 1 \]
	$4-1 = 3 \Rightarrow $ 1 wektor dołączony 2giego rzędu.

	\[ J =
		\begin{bmatrix}
			2 & 1 & 0 & 0\\
			0 & 2 & 1 & 0 \\
			0 & 0 & 2 & 0 \\
			0 & 0 & 0 & 3
		\end{bmatrix}
	\]
	\[ P =
		\begin{bmatrix}
			-2 & -1 & 3 \\
			0  & 1  & 2 \\
			1  & 2  & 4
		\end{bmatrix}
		\quad
		\begin{cases}
			-2x_2 - x_3 + 3x_4 &= x_1 \\
			x_3 + 2x_4 &= 0 \\
			x_2 + 2x_3 + 4x_4 &= 0
		\end{cases}
	\]
	\[ x_2 = 0 \]
	\[ x_3 = \frac 2 5 x_1 \]
	\[ x_4 = \frac 1 5 x_1 \]
	\[ x_1 \begin{bmatrix} 1 \\ 0 \\ -\frac 2 5 \\ \frac 1 5 \end{bmatrix} \]

	\[ (A-2E)^3 \cdot X = 0 \]
	\[ (A-2E)^2 \cdot X \neq 0 \]
	$ X = \begin{bmatrix} 0\\1\\0\\0\end{bmatrix} $ wektor dołączony 2go rzędu
	dla $\lambda = 2$, $Y = \begin{bmatrix}-2\\1\\1\\-1\end{bmatrix} $ 1go rzędu,
	$ Z = \begin{bmatrix}-6\\0\\0\\0\end{bmatrix} $

	\[ B = ( \ldots ) \]

	\section*{Zad. 34}
	\[ \lambda^2 - 6\lambda + 5 = (\lambda-1)(\lambda-5) \]
	\[ \lambda = 1 \lor \lambda = 5 \]
	\[ \begin{bmatrix} 1& 0\\ 0& 5 \end{bmatrix} \]
	\[ \begin{bmatrix} \frac 1 2 & 0 \\ 0 & \frac 5 2 \end{bmatrix} \]
	\[ \det(\frac 1 2 B -\lambda E) = (\lambda-\frac 1 2)(\lambda-\frac 5 2) =
		 \lambda^2 - 3\lambda + \frac 5 4 \]

	\section*{Zad. 36}
	\[ z= \overline{z^4} = \overline z ^ 4\]
	\[ |z| = |\overline z ^ 4 |= |z|^4 \]
	\[ |z|(|z|^3 - 1) = 0 \]
	\[ |z| = 1 \lor |z| = 0 \]
	Przy $|z|=0$ mamy  $z=0$, przy $|z|=1$:
	\[ z= \overline z ^ 4 \; |\cdot z^4 \]
	\[ z^5 = |z|^8 \]
	\[ z^5 = 1 \]
	Pierwiastki z 1 geometrycznie. %Znajdujemy pierwiastki sprzężone.

%	\section*{Zad. 44}
%	\[ E = \{ \begin{bmatrix}
	\section*{Zad. 32}
	\[ [A-\lambda E] \cdot v = 0 \]
	\[ \begin{bmatrix}
		x_1 - 2 & x_2 \\
		x_3 & x_4 - 2 \end{bmatrix}
		\cdot \begin{bmatrix} 1 \\ 4 \end{bmatrix}
		= (0,0) \]
	\[ \begin{bmatrix}
		x_1 - 1 & x_2 \\
		x_3 & x_4 - 1 \end{bmatrix}
		\cdot \begin{bmatrix} 1 \\ 3 \end{bmatrix}
		= (0,0) \]
	\[ x_1 -2 + 4x_2 = 0 \land x_1 - 1 + 3x+2 = 0 \]
	\[ x_3 + 4x_4 - 8 = 0 \land x_3 + 3 x_4 -3 =0 \]
	\[ x_1 = -2 \; x_2 = 1 \; x_3 = -12 \; x_4 = 5 \]

	łatwiej tak:
	\[ \phi((1,4)) = (2,8) \]
	\[ \phi((1,3)) = (1,3) \]
	korzystając z liniowości:
	\[ \phi((0,1)) = \phi((1,4)-(1,3)) = (2,8)-(1,3) = (1,5) \]
	\[ \phi((1,0)) = \phi((1,3)-3(0,1)) =  (1,3)-(3,15) = (-2,-12) \]
	\[ M_E^E (\phi) = \begin{bmatrix} -2 & 1 \\ -12&5 \end{bmatrix} \]
\end{document}

