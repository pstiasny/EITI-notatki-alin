\documentclass[a4paper,fleqn]{article}
\usepackage[utf8]{inputenc}
\usepackage[polish]{babel}
\usepackage{polski}
\usepackage{amsmath}
\usepackage{amssymb}

\usepackage{graphicx}
\usepackage{wrapfig}

\newcommand{\selim}[1][n]{\lim_{#1\to\infty}}
\newcommand{\Rls}{\mathbb{R}}
\newcommand{\Nats}{\mathbb{N}}
\newcommand{\Calk}{\mathbb{Z}}
\newcommand{\Complex}{\mathbb{C}}
\newcommand{\nNat}{n \in \Nats}
\newcommand{\xRl}{x \in \Rls}
\newcommand{\allNats}[1][n]{\forall_{#1 \in \Nats}}
\newcommand{\allRls}[1][x]{\forall_{#1 \in \Rls}}
\newcommand{\dla}{\text{ dla }}
\newcommand{\ddn}[1][n]{\text{ d.d.}#1\text{: }}
\renewcommand{\leq}{\leqslant}
\renewcommand{\geq}{\geqslant}
\newcommand{\sgn}{\operatorname{sgn}}
\newcommand{\im}{\operatorname{Im}}


\usepackage{fullpage}
\title{}
\author{}
\date{\today}
\begin{document}
	\section*{Zad. 1}
	Rozkład na ułamki proste

	\paragraph{a}
	\[ \frac{ 3x^2 - 6x - 4 }{ x^3 - 3x^2 - 4x + 12 } =
		\frac{ 3x^2 - 6x - 4 }{ (x-2)(x+2)(x-3) } \]
	\[ \frac{ 3x^2 - 6x - 4 }{ (x-2)(x+2)(x-3) } =
		\frac{A}{x-2} + \frac{B}{x-3} + \frac{C}{x+2} \]
	\[ 3x^2 - 6x - 4 = A(x-3)(x+2) + B(x-2)(x+2) + C(x-2)(x-3) \]
	\begin{align*}
		\dla x = 3:& \quad 5 = B(3-2)(3+2) = 5B \Rightarrow B = 1 \\
		x = 2:& \quad -4 = A(2-3)(2+2) = A \cdot (-1) \cdot 4 \Rightarrow A = 1 \\
		x = -2:& \quad 20 = C(-2-2)(-2-3) = C \cdot (-4) \cdot (-5) \Rightarrow C = 1
	\end{align*}

	\paragraph{b}
	\[ \frac{ x^4 + 3x^3 + 10x^2 + 9x + 26 }{ (x^2+x+5)^2 (x-1)^2} =
		\frac{A}{x-1} + \frac{B}{(x-1)^2} + \frac{Cx+D}{x^2+x+5} + \frac{Ex+F}{(x^2+x+5)^2} \]
	%\[ (x^2+x+5)^2 = x^4+x^2+25+2x^3+10x^2+10x \] %?
	rozkład licznika:
	\[ (x^4+2x^3+11x^2+10x+25) + \underbrace{(x^3-x^2-x+1)}_{x^2(x-1) - (x-1)} \]
	\[ \frac{ (x^2+x+5)^2 + (x-1)^2(x+1) }{ (x^2+x+5)^2 + (x-1)^2 } =
		\frac{1}{(x-1)^2} + \frac{x+1}{(x^2+x+5)^2} \]

	\paragraph{c}
	\[ \frac{ x^3+x^2+6x+6 }{ x^4+5x^3+7x^2+5x+6 } =
		\frac{ x^3+x^2+6x+6 }{ x^2(x^2+5x+6)+(x^2+5x+6) } =
		\frac{ x^3+x^2+6x+6 }{ (x^2+1)(x^2+5x+6) } =
		\frac{ x^2+5x+6+x(x^2+1) }{ (x^2+1)(x+2)(x+3) } \]
	\[ \frac{x}{ (x+2)(x+3) } = \frac{A}{x+2} + \frac{B}{x+3} =
		\frac{ A(x+3) + B(x+2) }{ (x+2)(x+3) } \]
	\begin{align*}
		\dla x = -2:& \quad A = -2 \\
		x = -3:& \quad B = 3
	\end{align*}

	\section*{Zad. 2}
	\paragraph{a}
	\[ \frac{ 2z^3 }{ z^4-1 } = \frac{ 2z^3 }{ (z^2-1)(z^2-(-1)) } =
		\frac{ 2z^3 }{ (z-1)(z+1)(z-j)(z+j) } =
		\frac{A}{z-1} + \frac{B}{z+1} + \frac{C}{z-j} + \frac{D}{z+j} \]
	\[ A(z+1)(z-j)(z+j) + B(z-1)(z-j)(z+j) + C(z+1)(z-1)(z+j) + D(z-1)(z+1)(z-j)
		= 2z^3 \]
	podstawiając wartości:
	\begin{align*}
		\dla z = 1:& \quad A = \frac 1 2 \\
		z = -1:& \quad B = \frac 1 2 \\
		z = j:& \quad C = \frac 1 2 \\
		z = -j:& \quad D = \frac 1 2
	\end{align*}
	a więc ostatecznie rozkład to:
	\[ \frac 1 2 \left(
		\frac{1}{z-1} + \frac{1}{z+1} + \frac{1}{z-j} + \frac{1}{z+j} \right) \]

	\paragraph{b}
	\[ \frac{ 2z^4 + 8z^2 + 32 }{ z(z^2 + 4)^2 } =
		\frac{ 2(z^4 + 8z^2 + 16) - 8z^2 }{ 2(z^2 + 4)^2 } =
		\frac 2 2 - \frac{8z}{(z^2+4)^2} \]
	\[ \frac{8z}{(z^2+4)^2} = \frac{A}{z+2j} + \frac{B}{(z+2j)^2} +
		\frac{C}{z-2j} + \frac{D}{(z-2j)^2} \]
	$A = 0$, $B=\frac 1 j$, $C = 0$, $D = - \frac 1 j$

\end{document}

